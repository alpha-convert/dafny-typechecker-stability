\RequirePackage{listings}
\RequirePackage{xcolor}
\RequirePackage{fvextra}


\colorlet{DFYgreen}{green!40!black}
\colorlet{DFYviolet}{violet}
\colorlet{DFYblue}{blue!50!black}
\colorlet{DFYorange}{orange!70!black}
\colorlet{DFYred}{red!50!black}
\colorlet{DFYgray}{black!70!white}
\colorlet{DFYteal}{teal!80!black}


\RequirePackage{xstring}
\RequirePackage{pgffor}
\RequirePackage{xparse}
\RequirePackage{etoolbox}
\RequirePackage{expl3}

%C
\colorlet{DFYLcommentColor}{DFYgreen}
\colorlet{DFYLnumberColor}{blue}
\colorlet{DFYLKWColor}{DFYblue}
\colorlet{DFYLKW2Color}{DFYgreen}
\colorlet{DFYLstringColor}{DFYred}
\colorlet{DFYLoperatorColor}{DFYred}
\colorlet{DFYLmiscKWColor}{DFYteal}
\colorlet{DFYLcheckerColor}{DFYteal}
%
\makeatletter


% Patch line number key to call line background macro
\lst@Key{numbers}{none}{%
	\def\lst@PlaceNumber{\lst@linebgrd}%
	\lstKV@SwitchCases{#1}%
	{none:\\%
		left:\def\lst@PlaceNumber{\llap{\normalfont
				\lst@numberstyle{\thelstnumber}\kern\lst@numbersep}\lst@linebgrd}\\%
		right:\def\lst@PlaceNumber{\rlap{\normalfont
				\kern\linewidth \kern\lst@numbersep
				\lst@numberstyle{\thelstnumber}}\lst@linebgrd}%
	}{\PackageError{Listings}{Numbers #1 unknown}\@ehc}}

% New keys
\lst@Key{linebackgroundcolor}{}{%
	\def\lst@linebgrdcolor{#1}%
}
\lst@Key{linebackgroundsep}{0pt}{%
	\def\lst@linebgrdsep{#1}%
}
\lst@Key{linebackgroundwidth}{\linewidth}{%
	\def\lst@linebgrdwidth{#1}%
}
\lst@Key{linebackgroundheight}{\ht\strutbox}{%
	\def\lst@linebgrdheight{#1}%
}
\lst@Key{linebackgrounddepth}{\dp\strutbox}{%
	\def\lst@linebgrddepth{#1}%
}
\lst@Key{linebackgroundcmd}{\color@block}{%
	\def\lst@linebgrdcmd{#1}%
}

% Line Background macro
\newcommand{\lst@linebgrd}{%
	\ifx\lst@linebgrdcolor\empty\else
	\rlap{%
		\lst@basicstyle
		\color{-.}% By default use the opposite (`-`) of the current color (`.`) as background
		\lst@linebgrdcolor{%
			\kern-\dimexpr\lst@linebgrdsep\relax%
			\lst@linebgrdcmd{\lst@linebgrdwidth}{\lst@linebgrdheight}{\lst@linebgrddepth}%
		}%
	}%
	\fi
}

\newcommand\storefirst{}
\newcommand\btLstHL[1]{\renewcommand\storefirst{#1}\btLstHLi}

\newcommand{\btLstHLi}[1][\storefirst]{%
	\ifnum\value{lstnumber}<\storefirst%
	\else%
	\ifnum\value{lstnumber}>#1%
	\else%
	\color{HLColor}%
	\fi%
	\fi%
}%

\lst@Key{DFYLnumberstyle}{}{\def\lst@DFYLnumberstyle{#1}}
\lst@Key{formatidstyle}{}{\def\lst@formatidstyle{#1}}
\lst@Key{checkerstyle}{}{\def\lst@checkerstyle{#1}}

\lst@Key{operators}{}{%
	\let\lst@operators\@empty
	\lst@for{#1}\do{%
		\lst@MakeActive{##1}%
		\lst@lExtend\lst@operators{%
			\expandafter\lst@CArg\lst@temp\relax\lst@CDef
			{}\lst@PrintOperator\@empty}}}
\lst@AddToHook{SelectCharTable}{\lst@operators}

\gdef\lst@PrintOperator#1\@empty{%
	\lst@XPrintToken
	\begingroup
	\lst@modetrue \lsthk@TextStyle
	\let\lst@ProcessDigit\lst@ProcessLetter
	\let\lst@ProcessOther\lst@ProcessLetter
	\lst@lettertrue
	#1%
	\lst@SaveToken
	\endgroup
	\lst@RestoreToken
	\global\let\lst@savedcurrstyle\lst@currstyle
	\ifnum\lst@mode=\the\lst@Pmode
	\let\lst@currstyle\lst@gkeywords@sty
	\fi
	\lst@Output
	\let\lst@currstyle\lst@savedcurrstyle}




% ``state variables''
\newif\ifinstring
\def\instringtrue{\global\let\ifinstring\iftrue}
\def\instringfalse{\global\let\ifinstring\iffalse}
\newif\ifinidentifier
\def\inidentifiertrue{\global\let\ifinidentifier\iftrue}
\def\inidentifierfalse{\global\let\ifinidentifier\iffalse}
\newif\ifinbrace
\def\inbracetrue{\global\let\ifinbrace\iftrue}
\def\inbracefalse{\global\let\ifinbrace\iffalse}
\newif\ifincomment
\def\incommenttrue{\global\let\ifincomment\iftrue}
\def\incommentfalse{\global\let\ifincomment\iffalse}
\newif\ifintick
\def\inticktrue{\global\let\ifintick\iftrue}
\def\intickfalse{\global\let\ifintick\iffalse}
\newif\ifDFYLparsenumbers
\lst@AddToHook{Output}{\@DFYLlcaddedToOutput}
\lst@AddToHook{EndGroup}{\inbracefalse\instringfalse\incommentfalse\intickfalse}
\lst@AddToHook{AfterBeginComment}{\incommenttrue}
% local variables
\newif\ifidentifierStartsByDigit@
\newif\ifidentifierStartsByPercent@
\newif\ifidentifierStartsByColumn@
\newif\ifidentifierStartsByColumn@
\def\splitfirstchar#1{\@splitfirstchar#1\@nil}
\def\@splitfirstchar#1#2\@nil{\gdef\@testChar{#1}}
\def\@testChar%
{%
	\ifDFYLparsenumbers
	% copy the first token in \the\lst@token to \@testChar
	\expandafter\splitfirstchar\expandafter{\the\lst@token}%
	\ifidentifierStartsByPercent@%
	\ifinstring%
	\def\lst@thestyle{\lst@formatidstyle}%
	\fi%
	\fi%
	%checker stuff
	\ifinbrace%
	\ifinstring%
	\def\lst@thestyle{\lst@stringstyle}%
	\else%
%	\ifidentifierStartsByColumn@%
	\def\lst@thestyle{\lst@checkerstyle}%
%	\else%
%	\fi%
	\fi%
	\else%
	\fi	%	
	% reset switch
	\identifierStartsByDigit@false%
	\identifierStartsByPercent@false%
	\identifierStartsByColumn@false%
	% equality test
	\IfInteger{\@testChar}{\identifierStartsByDigit@true}{}%
	\IfStrEq{\@testChar}{\%}{\identifierStartsByPercent@true}{}%
	\IfStrEq{\@testChar}{:}{\identifierStartsByColumn@true}{}%
	% processing the tests
	% numbers processing
	\ifidentifierStartsByDigit@%
	\ifnum\lst@mode=\the\lst@Pmode%
	\let\lst@thestyle=\lst@DFYLnumberstyle%
	\fi
	\fi
	% format string identifiers
	\ifidentifierStartsByPercent@%
	\ifinstring
	\def\lst@thestyle{\lst@formatidstyle}%
	\else
	\identifierStartsByPercent@false%
	\fi
	\fi
	\fi
}
\let\@DFYLlcaddedToOutput\@testChar


\makeatother


\lstdefinelanguage
{Dafny}
{
	alsoletter={0123456789?_'},
	showstringspaces = false,
	morestring = [s][\instringtrue\color{DFYLstringColor}]{"}{"},
	moredelim = *[s][\inbracetrue]{\{:}{\}},
	morecomment=[s]{/*}{*/},
	morecomment=[l]//,
	operators={&,|,:,=,+,!,-,>,<,\%},
	morekeywords=[0]{bool, char, int, int32, int64 iset, map, multiset, 
		nat, null, real, seq, set,
		string, Ref, usize, String,
		array, array1, array2, array3, array4, %arrayToken up to 4
		},
	morekeywords = [1]{++,+=,&&,!=,>=,<=,=,+,||,\%},
	morekeywords=[2]{abstract , as , assert , assume , break ,
		calc , case , class , codatatype , colemma ,
		constructor , copredicate , datatype , decreases ,
		default , else , ensures , exists , extends , false ,
		forall , free , fresh , function , ghost , if , imap , import ,
		in , include , inductive , invariant , iterator , label ,
		lemma , match , method , modifies , modify ,
		module , new , newtype , object ,
		old , opened , predicate , print , protected ,
		reads , refines , requires , return , returns , 
		static , then , this , trait , true , type ,
		var , where , while , yield , yields, opaque, reveal,
	    export, provides, reveals, by, for},  %additional keywords
    	%list taken from dafny reference manual, moved types to KW0
    morekeywords=[3]{Exception, U, Result, List, T, Box, 
    Option, RefL, MRef, MRefL,ArrayList,  
    int32, A, Iterator}, %types specific to this paper
    morekeywords = [4]{'l},
    keywordstyle=[3]\bfseries\color{DFYLKWColor},
	keywordstyle=[0]\bfseries\color{DFYLKWColor},
	keywordstyle=[2]\bfseries\color{DFYLKW2Color},
	commentstyle=\color{DFYLcommentColor},
	DFYLnumberstyle=\color{DFYLnumberColor},
	stringstyle=\color{DFYLstringColor},
	formatidstyle=\color{DFYLnumberColor},
	checkerstyle=\color{DFYLcheckerColor},
	keywordstyle=[1]\color{DFYLoperatorColor},
	keywordstyle=[4]\color{DFYLstringColor},
}

\lstnewenvironment{dafny}[1][]{
	\lstset{ %
		basicstyle=\ttfamily,
		breakatwhitespace=true,
		breaklines=true,
		extendedchars=true,
		keepspaces=true,
		showspaces=false,
		showstringspaces=false,
		showtabs=false,
		tabsize=2,
		language=Dafny,
		#1
	}\DFYLparsenumberstrue
}{\DFYLparsenumbersfalse}

\lstdefinestyle{dafnyS}
{
	language=Dafny,columns=fixed,
	basicstyle={\ttfamily},
}
